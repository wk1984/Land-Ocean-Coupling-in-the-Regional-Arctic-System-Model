\section{Introduction}
 
The impacts of global climate change are expected to be particularly intense in the Arctic region (IPCC, 2007).
In fact, significant changes are already being observed in Arctic surface temperature Rigor (Colony, & Martin, 1999), sea ice extent and thickness (e.g., Kwok, et al. 2009; Maslowski, et al., 2012), land ice (Gardner, et al., 2011), streamflow (Peterson et al, 2002; McClelland, et al., 2006; Dai, et al., 2009), and ocean salinity (Steele & Ermold, 2004; Morison, et al., 2012).
Although the overall observed changes are in line with the physical responses to global climate change indicated by both model results and theoretical arguments, the response of individual system components is not spatially or temporally uniform.
Moreover, important global and regional feedback mechanisms exist within the Arctic system with regard to sea and land ice cover, ocean heat budget, current and salinity, and atmospheric chemistry.

Within the Arctic system, the ocean and sea ice components are perhaps the most vulnerable to changes in climate and exert the strongest feedbacks on regional and local climate.
This combined with the unmistakable declines in observed sea ice cover and volume in the Arctic Ocean highlight the importance of gaining a better understanding of the processes that drive changes in sea ice.
Ocean and sea ice dynamics are strongly coupled with the atmosphere and land through the transfer of heat, momentum, and mass. Within RASM, care is taken to explicitly represent the full heat and momentum budgets within the atmosphere, ocean, and sea ice models.
For computational purposes, there is not a closed ocean boundary in RASM; rather, the exchange of mass between the ocean and the land/atmosphere is represented within the ocean and ice models by changes in salinity, which is a principal driver of sea ice development and buoyancy dynamics. 
  
  
  
  
  
  
  