\subsection{Oceanic freshwater flux, role in ocean and sea ice dynamics}

The seasonal freshwater flux from the land to the ocean plays an important role in coastal ocean hydrography and dynamics as well as to sea ice formation and melt.
Runoff from Arctic river basins is the primary source of buoyancy-driven coastal currents, such as the Alaska Coastal Current (ACC), Siberian Coastal Current, Norwegian Coastal Current, and East Greenland Coastal Current (e.g. Morison et al., 2000; Boyd et al., 2002; McGeehan & Maslowski, 2012).
Such currents redistribute both fresh water and heat, which locally play important roles in the shelf dynamics and shelf-basin interaction. Another important aspect is the effect of buoyancy and heat fluxes delivered by rivers on the onset of sea ice formation in winter and melt in spring/summer.
First, less salty water freezes at higher temperature, i.e. it does not have to be cooled as much as higher salinity water to freeze.
Thus, for a warming and freshening arctic, the onset of freezing in areas highly influenced by streamflow may be partially buffered against regional warming.
Second, heat delivered with runoff combined with the ice-albedo effect can make a big difference on the local retreat of sea ice from a shelf.
Finally, long-term river runoff maintains the surface freshwater layer in the Arctic Ocean and is a critical and necessary source required to balance freshwater export through Fram Strait and the Canadian Arctic Archipelago into the North Atlantic.

Recent work has focused on the development of a routing model to complete the hydrological cycle and represent the freshwater flux from the land surface to the Arctic Ocean.
This paper outlines the routing model implemented within RASM and some initial streamflow results.
Lastly, a set of science questions is laid out for future research using the RASM routing model.
  
  
  