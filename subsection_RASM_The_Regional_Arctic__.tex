\subsection{RASM}

The Regional Arctic System Model (RASM) is a fully coupled, high spatial and high temporal resolution regional climate model that has been recently developed under the support of the Department of Energy Earth System Modeling program.
Principle goals of the modeling project are to better understand the interaction between physical systems in the Arctic drainage basin, to advance understanding of past and present states of Arctic climate, and to improve seasonal to multi-decadal prediction capabilities of key Arctic indicators.
Existing model components, typically run off-line of one another are coupled using the Community Earth System Model (CESM) coupled model framework and the CPL7 flux coupler (Craig, 2012).
RASM utilizes the Weather Research Forecasting (WRF) model as the atmospheric component (Skamarock, et al., 2008), the Variable Infiltration Capacity (VIC) model as the land model (e.g. Liang, 1994, Gao, 2010), the Parallel Ocean Program (POP) as the ocean component (Smith & Gent, 2004), and the Los Alamos sea ice model (CICE) as the sea ice component (Bailey, et al., 2010).
Additional functionality, including component models to represent dynamic vegetation and land ice is currently being added.

\subsubsection{CICE}

\subsubsection{POP}

\subsubsection{VIC}

\subsubsection{WRF}

  
  